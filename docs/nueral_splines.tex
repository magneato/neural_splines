\documentclass[12pt]{article}
\usepackage[a4paper,margin=1in]{geometry}
\usepackage{amsmath,amssymb}
\usepackage{hyperref}
\usepackage{graphicx}
\usepackage{listings}
\usepackage{color}

% Define a light background for code listings
\definecolor{codebg}{rgb}{0.95,0.95,0.95}
\lstset{
  backgroundcolor=\color{codebg},
  basicstyle=\ttfamily\small,
  columns=fullflexible,
  frame=single,
  breaklines=true,
  language=Python,
  captionpos=b
}

\title{Neural Splines for Resource--Constrained Deep Learning}
\author{\textit{Prepared for the Neural Splines Project}}
\date{\today}

\begin{document}
\maketitle

\begin{abstract}
This technical note accompanies the \emph{Neural Splines} repository.  It distils the key theoretical
ideas behind spline--based weight compression and demonstrates how those ideas are realized in
code.  The goal is to provide researchers from diverse backgrounds with a concise yet
comprehensive exposition.  Readers are encouraged to explore the repository for full
implementations and examples.
\end{abstract}

\tableofcontents

\section{Introduction}

Neural networks are powerful function approximators, but their dense parameterization can be a
liability on devices with limited memory or compute.  Splines offer an elegant solution: by
representing the weight matrix with a grid of \emph{control points} and reconstructing the full
matrix via interpolation, we dramatically reduce the number of trainable parameters while
maintaining expressivity.  This technique is particularly appealing for scenarios where models
must run on edge devices or in environments with intermittent connectivity.

The original paper (see the attached Markdown document) situates neural splines within a broad
philosophical framework.  Here we focus on the practical aspects: defining spline layers,
training a simple classifier on the MNIST dataset, densifying the resulting network and
running inference with standard PyTorch.  All components in this repository are implemented
using only the core PyTorch library.

\section{Spline--Based Compression}

Consider a fully–connected layer with input dimension $d_{\text{in}}$ and output dimension
$d_{\text{out}}$.  A standard implementation stores a weight matrix $W\in\mathbb{R}^{d_{\text{out}}\times d_{\text{in}}}$
and a bias vector $b\in\mathbb{R}^{d_{\text{out}}}$, giving $d_{\text{out}}\times d_{\text{in}}+d_{\text{out}}$ parameters.
When $d_{\text{in}}$ and $d_{\text{out}}$ are large, this representation becomes cumbersome.

A spline layer stores instead a much smaller grid of control points.  For example, we can hold a
\(N\times N\) grid \(\{c_{i,j}\}\_{0\leq i,j<N}\) for the weight matrix and a vector of \(N\) control
points for the bias.  To compute the weight at arbitrary indices \((u,v)\), we map the indices
to a normalized coordinate in $[0,1]$ and apply bicubic interpolation.  Formally, for $u$ in
$\{0,\ldots,d_{\text{out}}-1\}$ and $v$ in $\{0,\ldots,d_{\text{in}}-1\}$, let

\[
t_u = \frac{u}{d_{\text{out}}-1}, \qquad t_v = \frac{v}{d_{\text{in}}-1}.
\]

Mapping $t_u$ and $t_v$ into the control–point grid yields indices \(i = t_u\,(N-3)\), $j = t_v\,(N-3)$,
with fractional parts \(\alpha, \beta \in [0,1]\).  The weight at $(u,v)$ is then computed via a
weighted sum of the sixteen neighbouring control points with cubic B–spline basis functions:

\[
W_{u,v} = \sum_{p=0}^3\sum_{q=0}^3 B_p(\alpha)\,B_q(\beta)\,c_{i+p,j+q},
\]

where $B_0,\dots,B_3$ are the cubic basis functions.  The bias term is obtained similarly from
a one–dimensional spline over its control points.  This mechanism enforces smoothness across
the virtual weight matrix and yields an enormous compression ratio: in our experiments we
replace tens of thousands of parameters with fewer than 100 control values.

\subsection{Control Point Optimization}

The control points themselves are learned via gradient descent.  During training we compute
virtual weights and biases on the fly, perform the forward and backward passes, and update
the control points.  The interpolation introduces a modest computational overhead but pays
dividends in memory savings.  Once training is complete, we can \emph{densify} the network by
explicitly evaluating the spline on every index and storing the resulting dense tensors.

\section{Reference Implementation}

The repository provides a reference implementation in `src/neural_spline.py`.  Below we
present the core of the spline–linear layer, focusing on clarity rather than performance.

\begin{lstlisting}[caption={Simplified implementation of a spline‐based linear layer.}]
import torch
import torch.nn as nn
import torch.nn.functional as F

class SplineLinear(nn.Module):
    def __init__(self, in_features: int, out_features: int, num_control: int = 6):
        super().__init__()
        # 2D control points for weights and 1D for bias
        self.weight_cp = nn.Parameter(torch.randn(num_control, num_control))
        self.bias_cp = nn.Parameter(torch.randn(num_control))
        self.in_features = in_features
        self.out_features = out_features
        self.num_control = num_control

    def interp2d(self, u_idx: torch.Tensor, v_idx: torch.Tensor) -> torch.Tensor:
        """Perform bicubic interpolation at discrete positions."""
        # Normalize to [0,1]
        t_u = u_idx.float() / (self.out_features - 1)
        t_v = v_idx.float() / (self.in_features - 1)
        # Map to control grid coordinates
        coords_u = t_u * (self.num_control - 3)
        coords_v = t_v * (self.num_control - 3)
        i = coords_u.floor().clamp(0, self.num_control - 4).long()
        j = coords_v.floor().clamp(0, self.num_control - 4).long()
        alpha = (coords_u - i).unsqueeze(-1)
        beta  = (coords_v - j).unsqueeze(-1)
        # Precompute basis functions B0–B3 for alpha and beta
        def B0(x): return ((1 - x) ** 3) / 6
        def B1(x): return (3*x**3 - 6*x**2 + 4) / 6
        def B2(x): return (-3*x**3 + 3*x**2 + 3*x + 1) / 6
        def B3(x): return (x ** 3) / 6
        Bs = [B0, B1, B2, B3]
        weight = 0
        # Sum over neighbourhood
        for p in range(4):
            for q in range(4):
                b_u = Bs[p](alpha)
                b_v = Bs[q](beta)
                cp = self.weight_cp[i + p, j + q]
                weight += (b_u * b_v) * cp
        return weight

    def forward(self, x: torch.Tensor) -> torch.Tensor:
        # Create index grids and interpolate
        idx_out = torch.arange(self.out_features, device=x.device)
        idx_in  = torch.arange(self.in_features, device=x.device)
        W = self.interp2d(idx_out[:, None], idx_in[None, :])
        # Bias via 1D cubic interpolation (similar logic)
        t = idx_out.float() / (self.out_features - 1)
        coords = t * (self.num_control - 3)
        k = coords.floor().clamp(0, self.num_control - 4).long()
        u = coords - k
        b = (
            ((1 - u)**3) / 6 * self.bias_cp[k] +
            ( (3*u**3 - 6*u**2 + 4) / 6) * self.bias_cp[k+1] +
            ((-3*u**3 + 3*u**2 + 3*u + 1) / 6) * self.bias_cp[k+2] +
            (u**3 / 6) * self.bias_cp[k+3]
        )
        return x @ W.t() + b

    def to_dense(self) -> nn.Linear:
        """Convert control points to a standard dense linear layer."""
        with torch.no_grad():
            idx_out = torch.arange(self.out_features)
            idx_in  = torch.arange(self.in_features)
            W = self.interp2d(idx_out[:, None], idx_in[None, :])
            t = idx_out.float() / (self.out_features - 1)
            coords = t * (self.num_control - 3)
            k = coords.floor().clamp(0, self.num_control - 4).long()
            u = coords - k
            b = (
                ((1 - u)**3) / 6 * self.bias_cp[k] +
                ( (3*u**3 - 6*u**2 + 4) / 6) * self.bias_cp[k+1] +
                ((-3*u**3 + 3*u**2 + 3*u + 1) / 6) * self.bias_cp[k+2] +
                (u**3 / 6) * self.bias_cp[k+3]
            )
        dense = nn.Linear(self.in_features, self.out_features)
        dense.weight.copy_(W)
        dense.bias.copy_(b)
        return dense
\end{lstlisting}

The `SplineLinear` class defines a weight grid and a bias spline.  The
\texttt{forward} method synthesizes dense weights and bias on demand, performs a matrix
multiplication, and adds the bias.  The `to\_dense` function evaluates the interpolants once
and packs the results into a conventional `nn.Linear`, which can then be used
in place of the spline layer during inference.

\section{Training on MNIST}

The script `src/train.py` trains a simple two–layer network on the MNIST digit dataset.  The
first layer uses spline compression; the second is a standard linear classifier.  During
training we report both the spline model’s accuracy and the accuracy of the densified model
obtained via `to_dense`.  A typical invocation is

\begin{lstlisting}[language=bash, caption={Training the spline network.}]
python3 src/train.py \
  --epochs 5 --batch-size 128 --cp 6 --hidden-size 256
\end{lstlisting}

On a modern CPU the training completes in a few minutes and achieves around 96 % test
accuracy.  The script writes checkpoint files for both the spline and densified models.

\section{Densification and Inference}

After training we often wish to deploy the network on systems where
on‑the‑fly interpolation is undesirable.  To this end we convert each
spline layer into a fixed dense layer using the `to\_dense` method and
assemble a dense multilayer perceptron.  This densified model no
longer references control points; its weights and biases are plain
tensors and can be executed with vanilla PyTorch.  The repository
contains `src/inference.py`, which reconstructs a dense network, loads
its state dictionary and computes the classification accuracy on the
MNIST test set.  A typical command line invocation is

\begin{lstlisting}[language=bash, caption={Running inference with the densified model.}]
python3 src/inference.py \
  --model-path checkpoints/dense_model.pth \
  --input-size 784 --hidden-size 256 --output-size 10
\end{lstlisting}

If you trained with different hidden sizes or control point counts you
should adjust the arguments accordingly.

\section{Conclusion}

Neural splines provide an intuitive yet powerful framework for compressing neural networks
without sacrificing performance.  By replacing thousands of individual weights with a small set
of interpolated control points, we obtain models that are compact, smooth, and efficient.  The
accompanying code and scripts in this project demonstrate how to implement, train and deploy
such networks using PyTorch.  We hope this concise exposition and the
well–commented code will serve as a useful starting point for researchers exploring
resource–constrained deep learning.

\end{document}
